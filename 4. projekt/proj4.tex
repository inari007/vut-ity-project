\documentclass[10pt,a4paper]{article}
\usepackage[utf8]{inputenc}
\usepackage[T1]{fontenc}
\usepackage[text={18cm, 24.5cm}, left=1.5cm, top=2cm]{geometry}
\usepackage[czech]{babel}
\bibliographystyle{czplain}
\usepackage{hyperref}

\begin{document}
\begin{titlepage}
    \begin{center}
        \textsc{\Huge Vysoké učení technické v Brně             
        \bigskip \\}
        \textsc{\huge Fakulta informačních technologií}
        \vspace{\stretch{0.382}}
        
        {\LARGE Typografie a publikování - 4. projekt \\
            \vspace{0.3em}
        \Huge Bibliografické citace}
        \vspace{\stretch{0.618}}
        
        {\Large \today \hfill 
        Zdeněk Dobeš}
    \end{center}
\end{titlepage}

\section*{Knihtisk}

Knihtisk je inovativní metoda 15. století v manufaktuře literárních kopií. Tento nový způsob výroby knih rapidně urychlil výrobní proces, čímž zapříčinil pokles tržních cen knih a rozšířil literaturu do širší populace v rámci mezí nižší čtenářské úrovně tehdejší doby \cite{Voit}. Za vznikem knihtisku stojí zlatník a brusič drahokamů Johannes Gutenberg \cite{JG}, jež jej sestrojil bez znalosti již existující obdobné Čínské technologie z 9. století, využívajíc historicky vůbec poprvé papír jako médium \cite{rep}. Jeho autorství však nebylo bezesporné, o zásluhy za toto dílo zápasilo hned několik tiskařů z okolních zemí \cite{MM}. Prvním knihtiskem masově prodkuvaným dílem byla Gutenbergova bible, která se svými 49 doposud existujícími výtisky ze 180 původních dosahuje k dnešnímu dni hodnoty 25-35 miliony dolarů \cite{history}.

\subsection*{Výroba}
Metody knihtisku se mohli aplikovat prostřednictvím tiskařského lisu. Toto zařízení, připisované stejnému autorovi, bylo inspirováno stroji na lisování vína a oliv \cite{Tisk_lis}. Kromě toho nesměly v tiskařské dílně chybět ani sazečské kasy, které sloužily jako zásobníky očištěných písmen, připravené k tisku \cite{Sazec}. Kromě vybavení bylo také nezbytné obdržet souhlas státní správy, neboť psané slovo bylo striktně regulováno vlivem církve, mnohdy ještě před samotným vydáním. Kvůli tisku zakázaných titulů bylo dokonce několik tiskařů i popraveno \cite{IF}.

\subsection*{Knihtisk v zahraničí}

Gutenberg se po dlouhou dobu snažil výrobní tajemství dílny utajit, ale po prohraném soudním sporu, jenž vedl k zabavení jeho dílny, museli učni vláčející znalostní břímě této manufakturní metody odejít, což mělo za následek rozšíření knihtisku do celé Evropy \cite{knihtisk}. Přístup jednotlivých zemí byl odlišný, Itálie se stala vedoucím producentem tišněných děl, zatímco v Anglii o metodě knihtisku vědělo po desetiletí pouze pár jedinců a její tajemství se dědilo pouze z učitele na učně \cite{eng}. V 16. století byl rozšířen i do zbytku světa, včetně zemí Severní Ameriky, Indie a dalekého Japonska \cite{knihtisk}.

\pagebreak

\bibliography{literature}

\end{document}